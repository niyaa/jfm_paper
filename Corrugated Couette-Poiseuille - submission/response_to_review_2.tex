\documentclass[a4paper,12pt]{article}
\usepackage[left=1.5cm,right=1.5cm,top=2cm,bottom=3.0cm]{geometry}

\usepackage{define}
\setlength{\parindent}{0pt}

\begin{document}
{\bf Response to comments provided by Referee 2}

Thank you for your comments. Below we copy these comments and provide our response. 

\vspace*{50pt}


{\bf Comments to the Author 

The paper corresponds to the scope of the Journal of Fluid Mechanics. The Couette-Poiseuille (CP) flow in the presence of the longitudinal grooves is considered numerically. The driving forces of the  flow are both the movement of the upper flat wall and pressure gradient imposed in the opposite direction. Analysis begins with characterization of stationary CP flow and followed by determination of conditions leading to the onset of linear destabilization. In the second part, the nonlinear solutions are obtained for the low values of the Reynolds numbers. Authors conclude that the distances required for the onset of measurable nonlinear effects of the instability can be decreased substantially. 

There are many publications devoted to the stability problem in channels with a wavy walls and the authors referenced to some of the publications in "the Introduction". The main new contribution of the manuscript is the flow configuration considered in the paper and corresponding changes of the boundary conditions. There are no new contributions regarding the governing equations, approaches or numerical methods to the stability analysis of the flows through a wavy channel. There are no qualitatively new results in the presented results of the manuscript.}

\vspace*{20pt}
We thank the reviewer for raising the issue of novelty, as it illustrates that the original version of the manuscript does not stress our intentions enough. We hope that the modified manuscript gives more highlight to what we perceive as the actual contribution of our work. The novelty presented in our work does not concern the instability or methods, neither we intend to centre our contribution around a slight modification of the flow configuration compared to what we investigated previously. Our intention is to present an idea of decreasing the advection velocity of an unstable, traveling wave. In our case it is an unstable mode that results from groove-imposed changes to the spanwise distribution of the streamwise velocity, that we previously studied in detail in the Poiseuille configuration. We wish to achieve the decrease in the advection velocity by application of the Couette-Poiseuille configuration. We in fact show that this modification can be applied to the point that the unstable wave becomes immobile leading to a possible change in the instability character from convective to a global one, which, in our opinion, is by itself unique. We also outline possible consequences of the results presented in the manuscript to possible experimental investigations of the considered instability type. We note that such attempts have failed so far, and in our opinion one of the reasons could have been the problem of the instability being convective and flushed out of the test domain. 

\vspace*{20pt}
{\bf Are the results and conclusions obtained in the manuscript sufficient for the publishing in JFM? On page 8 of the manuscript, the authors wrote "....retain the detailed parametric study of geometries for future consideration...". I suggest to carry out this computations in the manuscript.}
\vspace*{20pt}

The unstable wave considered in our work has already been extensively studied in the Poiseuille configuration by at least two groups. Ourselves, we published a rather detailed parametric study of this configuration, including nonlinear solutions it leads to for the Poiseuille configuration. For that reason, we do not think it is necessary to repeat those results. Consequently, the configuration selected for presentation in our manuscript leads to the onset of the hydrodynamic instability and allows to achieve decrease of the wave speed at moderate and low values of the Reynolds number ($Re<500$). This range of parameters is most interesting, since it is at low values of the Reynolds number that the supercritical character of the possible transition to the nonlinear state might be observed. In the modified version of the manuscript, we stress that the objective of our work is in examining possible decrease of the instability wave speed, down to the point that the instability character changes from convective to global and not the detailed parametric study of all possible corrugations. Finally, in our opinion such extension would blur the main point we are trying to make in the manuscript. 
\vspace*{20pt}

{\bf On page 7 of the manuscript, the authors wrote that it was enough to consider disturbances with delta=alpha. It is true if the flow is limited in the x-direction. There no bounds in the x-axis for the CP flow configuration considered in the paper. I suggest to add disturbances with delta<alpha in the manuscript analysis.}

\vspace*{20pt}
Considered configuration is periodic in the spanwise x- and streamwise z-direction. For the considered instability type it has been indicated in our previous work, that for configurations where the laminar flow becomes unstable at relatively low values of the Reynolds numbers, the spanwise wave number of the mode ($\delta$ in equation 2.7) remains locked to the wavenumber of the corrugation ($\alpha$ - wave number of the geometry in equation 2.1). This, we think is a consequence of the origin of the unstable wave that is thought to result from the spanwise variation of the streamwise velocity, which in turn results from channel’s geometry. We feel that the comment might steam from our insufficient explanation given in the text. This has been changed in the current version and the description highlights correspondence of respective wavenumbers. 
\vspace*{20pt}

{\bf In my opinion, the presented results are not enough for publication in the JFM.}

\end{document}
