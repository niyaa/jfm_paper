\documentclass[a4paper,12pt]{article}
\usepackage[left=1.5cm,right=1.5cm,top=2cm,bottom=3.0cm]{geometry}

\usepackage{define}
\setlength{\parindent}{0pt}

\begin{document}

{\bf Response to comments provided by Referee 1}

Thank you for your comments. Below we copy these comments and provide our response. 

\vspace*{50pt}
 

{\bf Comments to the Author 

The manuscript deals with analysis of dynamics of the Couette-Poiseuille flow with the stationary wall equipped with groves whose ridges are parallel to the flow direction. The analysis covers the primary states, their linear instability and nonlinear saturation states. The stated goal of the analysis is finding a combination of the Couette and Poiseuille flow components so that the instability waves become stationary with respect to the stationary wall. This should permit their experimental investigation as such waves would not travel out of the experimental apparatus making investigation of their growth and saturation states impossible. 

The concept presented in the manuscript is very well thought through, the analysis appears to be complete, interesting states have been identified. I recommend acceptance of this manuscript.}

\vspace*{20pt}
Thank you for your kind comment. 
\vspace*{20pt}

{\bf Introduction needs to be corrected which I view to be a small correction. Let me explain. The current text discusses instabilities, but it does not clearly state what the instability mechanisms are and where the challenge in experimental investigation is. The reader may be confused as to what the real intent of the analysis is. There are two instability mechanisms at work. One is driven by shear and has been investigated in the case of smooth walls by numerous authors – the manuscript quotes three papers with Klotz as the first author. This mechanism has been described in detail for the grooved wall by Moradi \& Floryan (2015) – reference used in the manuscript refers to Procedia which is a conference preprint. This reference needs to be replaced with   Moradi, H.V., Floryan J.M. 2014, Stability of flow in a channel with longitudinal grooves, J. Fluid Mech., vol. 757, pp. 613-648. The second mechanism is inviscid in nature, is driven by vorticity dynamics and inflection points in the spanwise distribution of the streamwise velocity component are the signature of its possible presence. This mechanism has been described in detail by Mohammadi et al (2015). This mechanism causes flow destabilization at low Re’s and this is the instability that the authors plan to study. Presence of grooves is essential for its activation as this is how the inflection points are formed. The authors need to clearly describe this situation and point out that they intend to investigate instability created by the inviscid, groove dependent mechanism. This information is missing from the Introduction which does not properly place the work in the overall body of knowledge.} 

\vspace*{20pt}
We modified the manuscript body in which we refer to the previous works on flow destabilization in the corrugated passage. It now clearly distinguishes the two types of unstable modes and informs the reader which one is of interest to this work. Also, the pointed-out inconsistency in the referred work has been corrected.  
\vspace*{20pt}

\newpage
{\bf The authors refer to the rate of strain in many places – what about stating that they refer to wall shear stress?}

\vspace*{20pt}
The reason we choose to refer to the rate of strain ($\gamma$) rather than the shear stress ($\tau$) is that in the case of the undisturbed, base flow its form remains independent of the Reynolds number, while shear stress in general does. Consequently, comparison with the saturated nonlinear states is, in our opinion, easier using the rate of strain, rather than shear stress. Alternatively, one may present the result as a product of shear stress and Reynolds number. 
\vspace*{20pt}

{\bf Line 437 – “result of nonlinear”}

\vspace*{20pt}
This has been corrected. 
\vspace*{20pt}

\end{document}
