\documentclass[lineno]{jfm}

\usepackage{graphicx}
% \usepackage{epstopdf, epsfig}
\usepackage{newtxtext}
\usepackage{newtxmath}
\usepackage{natbib}
\usepackage{hyperref}
\hypersetup{
    colorlinks = true,
    urlcolor   = blue,
    citecolor  = black,
}
\newtheorem{lemma}{Lemma}
\newtheorem{corollary}{Corollary}
\newcommand{\RomanNumeralCaps}[1]
\linenumbers

\DeclareMathOperator{\Var}{\it Var}

\usepackage{pbox}
\usepackage{multirow}
\usepackage{color, colortbl}
\definecolor{Gray}{gray}{0.65}
\newcommand{\RowColor}{\rowcolor{} }

% \usepackage[final]{changes}
\usepackage[markup]{changes}
% \usepackage{mdframed}

\title{From convective to global instability in corrugated Couette-Poiseuille flow}

\author
{
 N. Yadav\aff{1} \and
 S. W. Gepner\aff{1} \corresp{\email{stanislaw.gepner@pw.edu.pl}}
}

\affiliation{\aff{1}Warsaw University of Technology, Institute of Aeronautics and Applied Mechanics, Nowowiejska 24, 00-665 Warsaw, Poland}

\begin{document}
\maketitle

CP2.mp4:
Flow evolution throughout the linear amplification process captured for the $CP_2$ case. The process is started with small amplitude Gaussian noise perturbation and traced up to and past the onset of nonlinear interactions at saturation time $T_s$. Conditions correspond to $A=0.462$, $Re=300$ and $\beta=0.292$ ($Re_{cr}=296$, $\beta_{cr}=0.292$).
In the upper left corner, towards the negative $z$, iso-surfaces of the second invariant of the velocity gradient tensor taken at $80\%$ of the instantaneous maximum and coloured by streamwise vorticity also scaled by respective instantaneous maximum.
Towards the positive $z$-direction, streamwise velocity iso-surface at $w=0.2$ and coloured according to the spanwise velocity component $u$ along with iso-surfaces of the absolute second invariant of the velocity gradient at $Q=1.5\cdot10^{-4}$ (appearing around saturation time $Ts$).
The same is shown from the top perspective in the top-right part.
In the bottom scaled streamwise vorticity (left) and streamwise velocity (right).

\vspace{20pt}

CP3.mp4:
Flow evolution throughout the linear amplification process captured for the $CP_3$ case. The process is started with small amplitude Gaussian noise perturbation and traced up to and past the onset of nonlinear interactions at saturation time $T_s$. Conditions correspond to $A=0.455$, $Re =340$ and $\beta=0.281$ ($Re_{cr}=313$, $\beta_{cr}=0.281$). Composition is the same as in $CP_2$ case, with iso-surfaces of the absolute second invariant of the velocity gradient at $Q=5\cdot10^{-4}$ (also appearing around saturation time $T_s$).

\end{document}